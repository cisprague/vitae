%%%%%%%%%%%%%%%%%
% This is an example CV created using altacv.cls (v1.6.4, 13 Nov 2021) written by
% LianTze Lim (liantze@gmail.com), based on the
% Cv created by BusinessInsider at http://www.businessinsider.my/a-sample-resume-for-marissa-mayer-2016-7/?r=US&IR=T
%
%% It may be distributed and/or modified under the
%% conditions of the LaTeX Project Public License, either version 1.3
%% of this license or (at your option) any later version.
%% The latest version of this license is in
%%    http://www.latex-project.org/lppl.txt
%% and version 1.3 or later is part of all distributions of LaTeX
%% version 2003/12/01 or later.
%%%%%%%%%%%%%%%%

%% Use the "normalphoto" option if you want a normal photo instead of cropped to a circle
% \documentclass[10pt,a4paper,normalphoto]{altacv}

\documentclass[10pt,a4paper,ragged2e,withhyper]{altacv}

%% AltaCV uses the fontawesome5 package.
%% See http://texdoc.net/pkg/fontawesome5 for full list of symbols.

% Change the page layout if you need to
\geometry{left=1.25cm,right=1.25cm,top=1.5cm,bottom=1.5cm,columnsep=1.2cm}

% The paracol package lets you typeset columns of text in parallel
\usepackage{paracol}


% Change the font if you want to, depending on whether
% you're using pdflatex or xelatex/lualatex
\ifxetexorluatex
  % If using xelatex or lualatex:
  \setmainfont{Lato}
\else
  % If using pdflatex:
  \usepackage[default]{lato}
\fi

% Change the colours if you want to
\definecolor{VividPurple}{HTML}{417DC1}
\definecolor{SlateGrey}{HTML}{2E2E2E}
\definecolor{LightGrey}{HTML}{666666}
% \colorlet{name}{black}
% \colorlet{tagline}{PastelRed}
\colorlet{heading}{VividPurple}
\colorlet{headingrule}{VividPurple}
% \colorlet{subheading}{PastelRed}
\colorlet{accent}{VividPurple}
\colorlet{emphasis}{SlateGrey}
\colorlet{body}{LightGrey}

% Change some fonts, if necessary
% \renewcommand{\namefont}{\Huge\rmfamily\bfseries}
% \renewcommand{\personalinfofont}{\footnotesize}
% \renewcommand{\cvsectionfont}{\LARGE\rmfamily\bfseries}
% \renewcommand{\cvsubsectionfont}{\large\bfseries}

% Change the bullets for itemize and rating marker
% for \cvskill if you want to
\renewcommand{\itemmarker}{{\small\textbullet}}
\renewcommand{\ratingmarker}{\faCircle}

%% Use (and optionally edit if necessary) this .tex if you
%% want to use an author-year reference style like APA(6)
%% for your publication list
% When using APA6 if you need more author names to be listed
% because you're e.g. the 12th author, add apamaxprtauth=12
\usepackage[backend=biber,style=apa6,sorting=ydnt]{biblatex}
\defbibheading{pubtype}{\cvsubsection{#1}}
\renewcommand{\bibsetup}{\vspace*{-\baselineskip}}
\AtEveryBibitem{\makebox[\bibhang][l]{\itemmarker}}
\setlength{\bibitemsep}{0.25\baselineskip}
\setlength{\bibhang}{1.25em}


%% Use (and optionally edit if necessary) this .tex if you
%% want an originally numerical reference style like IEEE
%% for your publication list
% \usepackage[backend=biber,style=ieee,sorting=ydnt]{biblatex}
%% For removing numbering entirely when using a numeric style
\setlength{\bibhang}{1.25em}
\DeclareFieldFormat{labelnumberwidth}{\makebox[\bibhang][l]{\itemmarker}}
\setlength{\biblabelsep}{0pt}
\defbibheading{pubtype}{\cvsubsection{#1}}
\renewcommand{\bibsetup}{\vspace*{-\baselineskip}}


%% sample.bib contains your publications
\addbibresource{sample.bib}

\begin{document}
\name{Christopher Iliffe Sprague}
\tagline{Researcher in structured artificial intelligence}
% Cropped to square from https://en.wikipedia.org/wiki/Marissa_Mayer#/media/File:Marissa_Mayer_May_2014_(cropped).jpg, CC-BY 2.0
%% You can add multiple photos on the left or right
\photoR{2.5cm}{me}
% \photoL{2cm}{Yacht_High,Suitcase_High}
\personalinfo{%
  % Not all of these are required!
  % You can add your own with \printinfo{symbol}{detail}
  \email{sprague@kth.se}
  \phone{+46 073 080 7541}
  \mailaddress{Lindstedtsvägen 24, 114 28}
  \location{Stockholm, SE}
%   \homepage{marissamayr.tumblr.com}
\\
\printinfo{\faGraduationCap }{cisprague}[https://scholar.google.com/citations?user=3HsccBAAAAAJ&hl=en]
  \twitter{cisprague}
  \linkedin{cisprague}
  \github{cisprague} % I'm just making this up though.
%   \orcid{0000-0000-0000-0000} % Obviously making this up too.
  %% You can add your own arbitrary detail with
  %% \printinfo{symbol}{detail}[optional hyperlink prefix]
  % \printinfo{\faPaw}{Hey ho!}
  %% Or you can declare your own field with
  %% \NewInfoFiled{fieldname}{symbol}[optional hyperlink prefix] and use it:
  % \NewInfoField{gitlab}{\faGitlab}[https://gitlab.com/]
  % \gitlab{your_id}
	%%
  %% For services and platforms like Mastodon where there isn't a
  %% straightforward relation between the user ID/nickname and the hyperlink,
  %% you can use \printinfo directly e.g.
  % \printinfo{\faMastodon}{@username@instace}[https://instance.url/@username]
  %% But if you absolutely want to create new dedicated info fields for
  %% such platforms, then use \NewInfoField* with a star:
  % \NewInfoField*{mastodon}{\faMastodon}
  %% then you can use \mastodon, with TWO arguments where the 2nd argument is
  %% the full hyperlink.
  % \mastodon{@username@instance}{https://instance.url/@username}
}

\makecvheader

%% Depending on your tastes, you may want to make fonts of itemize environments slightly smaller
\AtBeginEnvironment{itemize}{\small}

%% Set the left/right column width ratio to 6:4.
\columnratio{0.6}

% Start a 2-column paracol. Both the left and right columns will automatically
% break across pages if things get too long.
\begin{paracol}{2}

\cvsection{Experience}

\cvevent{Researcher}{KTH Royal Institute of Technology | Robotics, Perception, and Learning Department}{Dec 2017 --  June 2022}{Stockholm, SE}
\begin{itemize}
\item Made contributions at the intersection of artificial intelligence, control theory, and machine learning (adv. Petter \"Ogren and John Folkesson).
\item Implemented planning and control algorithms on an in-house built autonomous underwater vehicle with ROS.
\item Supervised multiple M.Sc. students to the completion of their theses.
\end{itemize}

\divider

\cvevent{Autonomous Underwater Vehicle Assistant}{University of Tasmania | Institute for Marine and Antarctic Studies}{Dec 2019 -- Feb 2020}{Thwaites Glacier, Antarctica}
\begin{itemize}
\item Assisted in the deployment of the Nupiri Muka AUV in Western Antarctica
for under-ice data collection (adv. Peter King).
\item Assisted in the recover of oceanographic moorings. 
\end{itemize}

\divider

\cvevent{Researcher}{European Space Agency | Advanced Concepts Team}{Sep 2017 -- Nov 2017}{Noordwijk aan Zee, NL}
\begin{itemize}
  \item Made contributions to spacecraft trajectory optimisation with machine learning and optimal control (adv. Dario Izzo).
\end{itemize}

\divider

\cvevent{Researcher}{Japan Aerospace Exploration Agency | Institute of Space and Astronautical Science}{Jun 2017 --  Aug 2017}{Sagamihara, JP}
\begin{itemize}
  \item Researched machine learning for trajectory optimisation in the context of the lunar spacecraft mission EQUULEUS (adv. Yasuhiro Kawakatsu).
\end{itemize}


\divider

\cvevent{Learning Assistant}{Rensselaer Polytechnic Institute}{Aug 2016 -- May 2017}{Troy, NY, USA}
\begin{itemize}
  \item Held private consultation sessions and created a variety of workshops for study skills, time management, and stress management in order to promote academic excellence and encourage student involvement.
\end{itemize}

\divider

\cvevent{Software Engineer}{The Johns Hopkins University Applied Physics Laboratory}{Jun 2015 -- Aug 2015}{Laurel, MD, USA}
\begin{itemize}
  \item Produced targeted enhancements to the fault-protection systems of NASA's Solar Terrestrial Relations Observatory (adv. Dan Wilson and Kevin Balon).
  \item Updated the spacecrafts' testbeds to emulate their current operational modes.
\end{itemize}




% \cvsection{A Day of My Life}

% % Adapted from @Jake's answer from http://tex.stackexchange.com/a/82729/226
% % \wheelchart{outer radius}{inner radius}{
% % comma-separated list of value/text width/color/detail}
% % Some ad-hoc tweaking to adjust the labels so that they don't overlap
% \hspace*{-1em}  %% quick hack to move the wheelchart a bit left
% \wheelchart{1.5cm}{0.5cm}{%
%   10/13em/accent!30/Sleeping \& dreaming about work,
%   25/9em/accent!60/Public resolving issues with Yahoo!\ investors,
%   5/11em/accent!10/\footnotesize\\[1ex]New York \& San Francisco Ballet Jawbone board member,
%   20/11em/accent!40/Spending time with family,
%   5/8em/accent!20/\footnotesize Business development for Yahoo!\ after the Verizon acquisition,
%   30/9em/accent/Showing Yahoo!\ \mbox{employees} that their work has meaning,
%   5/8em/accent!20/Baking cupcakes
% }

% use ONLY \newpage if you want to force a page break for
% ONLY the currentc column
% \newpage

\cvsection{Publications}

\nocite{*}

\printbibliography[heading=pubtype,title={\printinfo{\faBook}{Books}},type=book]

% \divider

\printbibliography[heading=pubtype,title={\printinfo{\faFile*[regular]}{Journal Articles}}, type=article]

\divider

\printbibliography[heading=pubtype,title={\printinfo{\faUsers}{Conference Proceedings}},type=inproceedings]

%% Switch to the right column. This will now automatically move to the second
%% page if the content is too long.
\switchcolumn

% \cvsection{Life Philosophy}
% \begin{quote}
% ``To give anything less than your best is to sacrifice the gift.''
% \end{quote}

% \cvsection{Most Proud of}

% \cvachievement{\faTrophy}{Curiosity I had}{to explore the world and research.}

% \divider

% \cvachievement{\faHeartbeat}{Persistence \& Loyalty}{I showed despite the hard moments and my willingness to stay with Yahoo after the acquisition}

% \divider

% \cvachievement{\faChartLine}{Google's Growth}{from a hundred thousand searches per day to over a billion}

% \divider

% \cvachievement{\faFemale}{Inspiring women in tech}{Youngest CEO on Fortune's list of 50 most powerful women}

% \cvsection{Awards}

% \cvevent{East Asia and Pacific Summer Institute Fellowship}{National Science Foundation}{2017}{\$5400}
% % \begin{itemize}
% % \item Led the \$5 billion acquisition of the company with Verizon -- the entity which believed most in the immense value Yahoo!\ has created
% % \end{itemize}

% \divider

% \cvevent{NASA Fellowship}{The Johns Hopkins University Applied Physics Laboratory}{2015}{\$4000}
% % \begin{itemize}
% % \item Led the \$5 billion acquisition of the company with Verizon -- the entity which believed most in the immense value Yahoo!\ has created
% % \end{itemize}

\cvsection{Skills}

\textbf{Software} \\\smallskip
\cvtag{JAX}
\cvtag{PyTorch}
\cvtag{GPyTorch} 
\cvtag{Python}
\cvtag{ROS}

\divider\smallskip

\textbf{Theory} \\\smallskip
\cvtag{Optimal control}
\cvtag{Stability}
\cvtag{Order theory} \\
\cvtag{Hybrid dynamical systems}
\cvtag{Behavior trees}
\cvtag{Machine learning}
\cvtag{Physics-informed learning} \\
\cvtag{Fluid/structural mechanics}






\cvsection{Education}

\cvevent{Ph.D.\ in Robotics}{KTH Royal Institute of Technology}{Dec 2017 -- June 2022 (expected)}{}
\begin{itemize}
  \item Project on Robust, flexible and transparent mission planning and execution for autonomous underwater vehicles.
  \item Funded by Swedish Maritime Robotics Centre (SMaRC).
\end{itemize}

\divider

\cvevent{M.S.\ in Aerospace Engineering}{Rensselaer Polytechnic Institute}{May 2016 -- May 2017}{}
\begin{itemize}
  \item Magna Cum Laude honours (adv. Kurt Anderson).
\end{itemize}

\divider

\cvevent{B.S.\ in Aerospace Engineering}{Rensselaer Polytechnic Institute}{Aug 2013 -- May 2016}{}
\begin{itemize}
  \item Cum Laude honours.
\end{itemize}

\cvsection{Grants}

\cvevent{JSPS Summer Program}{Japan Society for the promotion of Science}{2017}{\textyen 692,500}

\cvevent{East Asia and Pacific Summer Institute Fellowship}{National Science Foundation}{2017}{\$5,400}
% \begin{itemize}
% \item Led the \$5 billion acquisition of the company with Verizon -- the entity which believed most in the immense value Yahoo!\ has created
% \end{itemize}
\divider

\cvevent{Congress-Bundestag Youth Exchange (CBYX) for Young Professionals}{German Bundestag and U.S. Department of State}{2015}{Declined}

\divider

\cvevent{NASA Fellowship}{The Henry Foundation, Inc.}{2015}{\$4000}
% \begin{itemize}
% \item Led the \$5 billion acquisition of the company with Verizon -- the entity which believed most in the immense value Yahoo!\ has created
% \end{itemize}

\cvsection{Languages}
\cvskill{English}{5}

\cvskill{Swedish}{2}
% \divider

\cvskill{Spanish}{1}
% \divider

% \newpage

\cvsection{Referees}

% \cvref{name}{email}{mailing address}
\cvref{Prof.\ Petter \"Ogren}{KTH Royal Institute of Technology}{petter@kth.se}
{Address Line 1\\Address line 2}

\divider

\cvref{Prof.\ John Folkesson}{KTH Royal Institute of Technology}{john@kth.se}
{Address Line 1\\Address line 2}

\end{paracol}

\end{document}
